% ---------------------------------------------------------------------
% Podstawowe ustawienia i pakiety
% ---------------------------------------------------------------------
\RequirePackage[l2tabu, orthodox]{nag} % Wykrywa przestarzałe i niewłaściwe
% sposoby używania LaTeXa. Więcej jest w l2tabu English version.
\documentclass[a4paper,11pt]{article}
% {rozmiar papieru, rozmiar fontu}[klasa dokumentu]
\usepackage[MeX]{polski} % Polonizacja LaTeXa, bez niej będzie pracował
% w języku angielskim.
\usepackage[utf8]{inputenc} % Włączenie kodowania UTF-8, co daje dostęp
% do polskich znaków.
\usepackage{lmodern} % Wprowadza fonty Latin Modern.
\usepackage[T1]{fontenc} % Potrzebne do używania fontów Latin Modern.



% ------------------------------
% Podstawowe pakiety (niezwiązane z ustawieniami języka)
% ------------------------------
\usepackage{microtype} % Twierdzi, że poprawi rozmiar odstępów w tekście.
\usepackage{graphicx} % Wprowadza bardzo potrzebne komendy do wstawiania
% grafiki.
\usepackage{verbatim} % Poprawia otoczenie VERBATIME.
\usepackage{textcomp} % Dodaje takie symbole jak stopnie Celsiusa,
% wprowadzane bezpośrednio w tekście.
\usepackage{vmargin} % Pozwala na prostą kontrolę rozmiaru marginesów,
% za pomocą komend poniżej. Rozmiar odstępów jest mierzony w calach.
% ------------------------------
% MARGINS
% ------------------------------
\setmarginsrb
{ 0.7in}  % left margin
{ 0.6in}  % top margin
{ 0.7in}  % right margin
{ 0.8in}  % bottom margin
{  20pt}  % head height
{0.25in}  % head sep
{   9pt}  % foot height
{ 0.3in}  % foot sep



% ------------------------------
% Często przydatne pakiety
% ------------------------------
% \usepackage{csquotes} % Pozwala w prosty sposób wstawiać cytaty do tekstu.
\usepackage{xcolor} % Pozwala używać kolorowych czcionek (zapewne dużo
% więcej, ale ja nie potrafię nic o tym powiedzieć).



% ------------------------------
% Pakiety do tekstów z nauk przyrodniczych
% ------------------------------
\let\lll\undefined % Amsmath gryzie się z językiem pakietami do języka
% polskiego, bo oba definiują komendę \lll. Aby rozwiązać ten problem
% oddefiniowuję tę komendę, ale może tym samym pozbywam się dużego Ł.
\usepackage[intlimits]{amsmath} % Podstawowe wsparcie od American
% Mathematical Society (w skrócie AMS)
\usepackage{amsfonts, amssymb, amscd, amsthm} % Dalsze wsparcie od AMS
% \usepackage{siunitx} % Dla prostszego pisania jednostek fizycznych
\usepackage{upgreek} % Ładniejsze greckie litery
% Przykładowa składnia: pi = \uppi
\usepackage{slashed} % Pozwala w prosty sposób pisać slash Feynmana.
% \usepackage{calrsfs} % Zmienia czcionkę kaligraficzną w \mathcal
% na ładniejszą. Może w innych miejscach robi to samo, ale o tym nic
% nie wiem.



% ---------------
% Wspaniały pakiet PGF/TikZ
% ---------------
\usepackage{tikz}

% Włączenie konkretnych bibliotek pakietu TikZ.
\usetikzlibrary{decorations.markings}

% Picsy TikZa
\usepackage{TikZPics}

% Dodatkowe style TikZa
\usepackage{TikZStyles}



% ------------------------------
% Tworzenie środowisk (?) „Twierdzenie”, „Definicja”, „Lemat”, etc.
% ------------------------------
% Komenda wprowadzająca otoczenie „theorem” do pisania twierdzeń
% matematycznych.
\newtheorem{theorem}{Twierdzenie}
% Analogicznie jak powyżej
\newtheorem{definition}{Definicja}
\newtheorem{corollary}{Wniosek}



% ---------------------------------------
% Pakiety napisane przez użytkownika.
% Mają być w tym samym katalogu to ten plik .tex
% ---------------------------------------
\usepackage{latexgeneralcommands}
% \usepackage{mathcommands}
% \newcommand{\conca}{\textrm{conca}}




% ---------------------------------------------------------------------
% Dodatkowe ustawienia dla języka polskiego
% ---------------------------------------------------------------------
\renewcommand{\thesection}{\arabic{section}.}
% Kropki po numerach rozdziału (polski zwyczaj topograficzny)
\renewcommand{\thesubsection}{\thesection\arabic{subsection}}
% Brak kropki po numerach podrozdziału



% ------------------------------
% Ustawienia różnych parametrów tekstu
% ------------------------------
\renewcommand{\baselinestretch}{1.1}

% Ustawienie szerokości odstępów między wierszami w tabelach.
\renewcommand{\arraystretch}{1.4}





% ------------------------------
% Pakiet „hyperref”
% Polecano by umieszczać go na końcu preambuły.
% ------------------------------
\usepackage{hyperref} % Pozwala tworzyć hiperlinki i zamienia odwołania
% do bibliografii na hiperlinki.










% ---------------------------------------------------------------------
% Tytuł i autor tekstu
\title{Matematyka, rysunki i~wykresy}

\author{Kamil Ziemian}


% \date{}
% ---------------------------------------------------------------------










% ####################################################################
\begin{document}
% ####################################################################





% ######################################
\maketitle % Tytuł całego tekstu
% ######################################



% ##################
\begin{center}

  \begin{tabular}{|c|c|}
    \hline
    \multicolumn{2}{|c|}{Wartości liczb} \\
    \hline
    $\pi = 3.1415...$ & $\sqrt{ \pi } = 1.7724...$ \\
    $e = 2.7182...$ & $\sqrt{ e } = 1.6487...$ \\
    $\phi = 1.6180...$ & $\sqrt{ \phi } = 1.2720...$ \\
    $\gamma = 0.5772...$ & $\sqrt{ \gamma } = 0.7597...$ \\
    \hline
  \end{tabular}

\end{center}
% ##################





% ##################
\begin{center}

  \begin{tabular}{|c|c|}
    \hline
    \multicolumn{2}{|c|}{Wartości liczb} \\
    \hline
    $\sqrt{ 1 } = 1$ & $\ln( 1 ) = 0$ \\
    $\sqrt{ 2 } = 1.4142...$ & $\ln( 2 ) = 0.6931...$ \\
    $\sqrt{ 3 } = 1.7320...$ & $\ln( 3 ) = 1.0986...$ \\
    $\sqrt{ 4 } = 2$ & $\ln( 4 ) = 1.3862...$ \\
    $\sqrt{ 5 } = 2.2360...$ & $\ln( 5 ) = 1.6094...$ \\
    $\sqrt{ 6 } = 2.4494...$ & $\ln( 6 ) = 1.7917...$ \\
    $\sqrt{ 7 } = 2.6457...$ & $\ln( 7 ) = 1.9459...$ \\
    $\sqrt{ 8 } = 2.8284...$ & $\ln( 8 ) = 2.0794...$ \\
    $\sqrt{ 9 } = 3$ & $\ln( 9 ) = 2.1972...$ \\
    $\sqrt{ 10 } = 3.1622...$ & $\ln( 10 ) = 2.3025...$ \\
    \hline
  \end{tabular}

  % \begin{tabular}{|c|c|}
  %   \hline
  %   Nazwa liczby & Wartość & Nazwa liczby & Wartość \\
  %   \hline
  %   $\sqrt{ 1 }$ & 1 & $\ln( 1 )$ & $0$ \\
  %   $\sqrt{ 2 }$ & 1.4142... & & \\
  %   $\sqrt{ 3 }$ & 1.7320... & & \\
  %   $\sqrt{ 4 }$ & 2 & & \\
  %   $\sqrt{ 5 }$ & 2.2360... \\
  %   $\sqrt{ 6 }$ & 2.4494... \\
  %   $\sqrt{ 7 }$ & 2.6457... \\
  %   $\sqrt{ 8 }$ & 2.8284... \\
  %   $\sqrt{ 9 }$ & 3 \\
  %   $\sqrt{ 10 }$ & 3.1622... \\
  %   \hline
  % \end{tabular}

\end{center}
% ##################





















% ##################
\begin{figure}

  \centering

  \label{fig:Rysunki-i-wykresy-01}


  \begin{tikzpicture}

    \draw[axis arrow thin] (-1,0) -- (10,0);

    \node at (9.7,-0.4) {$x$};



    \pic at (0,0) {tick x axis thin 1};

    \node at (0,-0.4) {$0$};


    \pic at (1,0) {tick x axis thin 2};

    \node at (1,-0.4) {$0.5$};


    \pic at (2,0) {tick x axis thin 1};

    \node at (2,-0.4) {$1$};


    \pic at (3,0) {tick x axis thin 2};

    \node at (3,-0.4) {$1.5$};


    \pic at (4,0) {tick x axis thin 1};

    \node at (4,-0.4) {$2$};


    \pic at (5,0) {tick x axis thin 2};

    \node at (5,-0.4) {$2.5$};


    \pic at (6,0) {tick x axis thin 1};

    \node at (6,-0.4) {$3$};


    \pic at (7,0) {tick x axis thin 2};

    \node at (7,-0.4) {$3.5$};


    \pic at (8,0) {tick x axis thin 1};

    \node at (8,-0.4) {$4$};





    \pic at (1.154,0) {point};

    \node at (1.154,0.3) {$\gamma$};



    \pic at (3.236,0) {point};

    \node (Golden ratio) at (3.236,0.35) {$\phi$};



    \pic at (5.436,0) {point};

    \node at (5.436,0.3) {$e$};



    \pic at (6.282,0) {point};

    \node at (6.282,0.3) {$\pi$};

  \end{tikzpicture}


  \caption{Położenie różnych stałych liczbowych na osi rzeczywistej.}


\end{figure}
% ##################





% ##################
\begin{figure}

  \centering

  \label{fig:Rysunki-i-wykresy-02}


  \begin{tikzpicture}

    \draw[axis arrow thin] (-1,0) -- (10,0);

    \node at (9.7,-0.4) {$x$};



    \pic at (0,0) {tick x axis thin 1};

    \node at (0,-0.4) {$0$};


    \pic at (1,0) {tick x axis thin 2};

    \node at (1,-0.4) {$0.5$};


    \pic at (2,0) {tick x axis thin 1};

    \node at (2,-0.4) {$1$};


    \pic at (3,0) {tick x axis thin 2};

    \node at (3,-0.4) {$1.5$};


    \pic at (4,0) {tick x axis thin 1};

    \node at (4,-0.4) {$2$};


    \pic at (5,0) {tick x axis thin 2};

    \node at (5,-0.4) {$2.5$};


    \pic at (6,0) {tick x axis thin 1};

    \node at (6,-0.4) {$3$};


    \pic at (7,0) {tick x axis thin 2};

    \node at (7,-0.4) {$3.5$};


    \pic at (8,0) {tick x axis thin 1};

    \node at (8,-0.4) {$4$};




    \pic at (2,0) {point};

    \node at (1.9,0.35) {$\sqrt{ 1 }$};



    \pic at (2.828,0) {point};

    \node (Sqrt2) at (2.75,0.35) {$\sqrt{ 2 }$};



    \pic at (3.464,0) {point};

    \node (Sqrt3) at (3.4,0.35) {$\sqrt{ 3 }$};



    \pic at (4,0) {point};

    \node (Sqrt4) at (4,1.5) {$\sqrt{ 4 }$};

    \draw[pointing arrow thin 1] (Sqrt4) -- (4,0.2);



    \pic at (4.472,0) {point};

    \node (Sqrt5) at (4.7,1.5) {$\sqrt{ 5 }$};

    \draw[pointing arrow thin 1] (Sqrt5) -- (4.5,0.2);



    \pic at (4.898,0) {point};

    \node (Sqrt6) at (5.4, 1.5) {$\sqrt{ 6 }$};

    \draw[pointing arrow thin 1] (Sqrt6) -- (4.95,0.2);



    \pic at (5.292,0) {point};

    \node (Sqrt7) at (6.1,1.5) {$\sqrt{ 7 }$};

    \draw[pointing arrow thin 1] (Sqrt7) -- (5.35,0.2);



    \pic at (5.656,0) {point};

    \node (Sqrt8) at (7,1.5) {$\sqrt{ 8 }$};

    \draw[pointing arrow thin 1] (Sqrt8) -- (5.75, 0.2);



    \pic at (6,0) {point};

    \node (Sqrt9) at (7.9,1.5) {$\sqrt{ 9 }$};

    \draw[pointing arrow thin 1] (Sqrt9) -- (6.1,0.2);



    \pic at (6.324,0) {point};

    \node (Sqrt10) at (9,1.5) {$\sqrt{ 10 }$};

    \draw[pointing arrow thin 1] (Sqrt10) -- (6.5,0.15);

  \end{tikzpicture}


  \caption{Położenie pierwiastków z~najmniejszych liczb naturalnych
    na~osi rzeczywistej.}


\end{figure}
% ##################





% ##################
\begin{figure}

  \centering

  \label{fig:Rysunki-i-wykresy-03}


  \begin{tikzpicture}

    \draw[axis arrow thin] (-1,0) -- (7,0);

    \node at (6.7,-0.4) {$x$};



    \pic at (0,0) {tick x axis thin 1};

    \node at (0,-0.4) {$0$};


    \pic at (1,0) {tick x axis thin 2};

    \node at (1,-0.4) {$0.5$};


    \pic at (2,0) {tick x axis thin 1};

    \node at (2,-0.4) {$1$};


    \pic at (3,0) {tick x axis thin 2};

    \node at (3,-0.4) {$1.5$};


    \pic at (4,0) {tick x axis thin 1};

    \node at (4,-0.4) {$2$};


    \pic at (5,0) {tick x axis thin 2};

    \node at (5,-0.4) {$2.5$};


    \pic at (6,0) {tick x axis thin 1};

    \node at (6,-0.4) {$3$};





    \pic at (1.518,0) {point};

    \node at (1.5,0.4) {$\sqrt{ \gamma }$};



    \pic at (2.544,0) {point};

    \node at (2.5,0.4) {$\sqrt{ \phi }$};



    \pic at (3.296,0) {point};

    \node (SqrtE) at (3.25,1.2) {$\sqrt{ e }$};

    \draw[pointing arrow thin 1] (SqrtE) -- (3.3,0.2);



    \pic at (3.5448,0) {point};

    \node (SqrtPi) at (4.2,1.2) {$\sqrt{ \pi }$};

    \draw[pointing arrow thin 1] (SqrtPi) -- (3.625,0.2);

  \end{tikzpicture}


  \caption{Położenie pierwiastków z~różnych stałych liczbowych
    na~osi rzeczywistej.}


\end{figure}
% ##################





% ##################
\begin{figure}

  \centering

  \label{fig:Rysunki-i-wykresy-04}


  \begin{tikzpicture}

    \draw[axis arrow thin] (-1,0) -- (7,0);

    \node at (6.7,-0.4) {$x$};



    \pic at (0,0) {tick x axis thin 1};

    \node at (0,-0.4) {$0$};


    \pic at (1,0) {tick x axis thin 2};

    \node at (1,-0.4) {$0.5$};


    \pic at (2,0) {tick x axis thin 1};

    \node at (2,-0.4) {$1$};


    \pic at (3,0) {tick x axis thin 2};

    \node at (3,-0.4) {$1.5$};


    \pic at (4,0) {tick x axis thin 1};

    \node at (4,-0.4) {$2$};


    \pic at (5,0) {tick x axis thin 2};

    \node at (5,-0.4) {$2.5$};


    \pic at (6,0) {tick x axis thin 1};

    \node at (6,-0.4) {$3$};





    \pic at (0,0) {point};

    \node (ln1) at (0,1.5) {$\ln( 1 )$};

    \draw[pointing arrow thin 1] (ln1) -- (0,0.2);



    \pic at (1.386,0) {point};

    \node (ln2) at (1.1,1.5) {$\ln( 2 )$};

    \draw[pointing arrow thin 1] (ln2) -- (1.35,0.2);



    \pic at (2.196,0) {point};

    \node (ln3) at (2.19,1.5) {$\ln( 3 )$};

    \draw[pointing arrow thin 1] (ln3) -- (2.195,0.2);



    \pic at (2.772,0) {point};

    \node (ln4) at (3.3,1.5) {$\ln( 4 )$};

    \draw[pointing arrow thin 1] (ln4) -- (2.825,0.2);




    \pic at (3.218,0) {point};

    \node (ln5) at (4.4,1.5) {$\ln( 5 )$};

    \draw[pointing arrow thin 1] (ln5) -- (3.35,0.2);



    \pic at (3.582,0) {point};

    \node (ln6) at (5.5,1.5) {$\ln( 6 )$};

    \draw[pointing arrow thin 1] (ln6) -- (3.7,0.2);



    \pic at (3.89,0) {point};

    \node (ln7) at (3.1, -1.5) {$\ln( 7 )$};

    \draw[pointing arrow thin 1] (ln7) -- (3.8,-0.2);



    \pic at (4.158,0) {point};

    \node (ln8) at (5,-1.5) {$\ln( 8 )$};

    \draw[pointing arrow thin 1] (ln8) -- (4.25,-0.2);



    \pic at (4.394,0) {point};

    \node (ln9) at (6.6,1.5) {$\ln( 9 )$};

    \draw[pointing arrow thin 1] (ln9) -- (4.55,0.2);



    \pic at (4.604,0) {point};

    \node (ln10) at (7.9,1.5) {$\ln( 10 )}$};

    \draw[pointing arrow thin 1] (ln10) -- (4.775,0.125);

  \end{tikzpicture}


  \caption{Położenie logarytmów naturalnych z~najmniejszych liczb
    naturalnych na~osi rzeczywistej.}


\end{figure}
% ##################





% ############################

% Koniec dokumentu
\end{document}