% ------------------------------------------------------------------------------------------------------------------
% Basic configuration and packages
% ------------------------------------------------------------------------------------------------------------------
% Package for discovering wrong and outdated usage of LaTeX.
% More information to be found in l2tabu English version.
\RequirePackage[l2tabu, orthodox]{nag}
% Class of LaTeX document: {size of paper, size of font}[document class]
\documentclass[a4paper,11pt]{article}



% ------------------------------------------------------
% Packages not tied to particular normal language
% ------------------------------------------------------
% This package should improved spaces in the text
\usepackage{microtype}
% Add few important symbols, like text Celcius degree
\usepackage{textcomp}



% ------------------------------------------------------
% Polonization of LaTeX document
% ------------------------------------------------------
% Basic polonization of the text
\usepackage[MeX]{polski}
% Switching on UTF-8 encoding
\usepackage[utf8]{inputenc}
% Adding font Latin Modern
\usepackage{lmodern}
% Package is need for fonts Latin Modern
\usepackage[T1]{fontenc}



% ------------------------------------------------------
% Setting margins
% ------------------------------------------------------
\usepackage[a4paper, total={14cm, 25cm}]{geometry}



% ------------------------------------------------------
% Setting vertical spaces in the text
% ------------------------------------------------------
% Setting space between lines
\renewcommand{\baselinestretch}{1.1}

% Setting space between lines in tables
\renewcommand{\arraystretch}{1.4}



% ------------------------------------------------------
% Packages for scientific papers
% ------------------------------------------------------
% Switching off \lll symbol, that I guess is representing letter "Ł"
% It collide with `amsmath' package's command with the same name
\let\lll\undefined
% Basic package from American Mathematical Society (AMS)
\usepackage[intlimits]{amsmath}
% Equations are numbered separately in every section
\numberwithin{equation}{section}

% Other very useful packages from AMS
\usepackage{amsfonts}
\usepackage{amssymb}
\usepackage{amscd}
\usepackage{amsthm}

% Package with better looking calligraphy fonts
\usepackage{calrsfs}

% Package with better looking greek letters
% Example of use: pi -> \uppi
\usepackage{upgreek}
% Improving look of lambda letter
\let\oldlambda\lambda
\renewcommand{\lambda}{\uplambda}




% ------------------------------------------------------
% BibLaTeX
% ------------------------------------------------------
% Package biblatex, with biber as its backend, allow us to handle
% bibliography entries that use Unicode symbols outside ASCII
% \usepackage[
% language=polish,
% backend=biber,
% style=alphabetic,
% url=false,
% eprint=true,
% ]{biblatex}

% \addbibresource{Logika-i-teoria-mnogości-Bibliography.bib}





% ------------------------------------------------------
% Defining new environments (?)
% ------------------------------------------------------
% Defining enviroment "Wniosek"
% \newtheorem{corollary}{Wniosek}
% \newtheorem{definition}{Definicja}
% \newtheorem{theorem}{Twierdzenie}





% ------------------------------------------------------
% Wonderful package PGF/TikZ
% ------------------------------------------------------
\usepackage{tikz}

% Loding TikZ libraries

% Library for positioning nodes
\usetikzlibrary{positioning}

% Styles for arrows
\usepackage{./Local-packages/PGF-TikZ-Arrows-styles}

% Node and pics for drawing charts
\usepackage{./Local-packages/PGF-TikZ-Chart-nodes-and-pics}




% ------------------------------------------------------
% Local packages
% ------------------------------------------------------
% Local configuration of this particular file
\usepackage{./Local-packages/local-settings}

% Package containing various command useful for working with a text
% \usepackage{general-commands}
% Package containing commands and other code useful for working with
% mathematical text
% \usepackage{math-commands}





% ------------------------------------------------------
% Package "hyperref"
% They advised to put it on the end of preambule
% ------------------------------------------------------
% It allows you to use hyperlinks in the text
\usepackage{hyperref}










% ------------------------------------------------------------------------------------------------------------------
% Title and author of the text
\title{Wykresy funkcji rzeczywistych}

\author{Kamil Ziemian \\
  \email}


% \date{}
% ------------------------------------------------------------------------------------------------------------------










% ####################################################################
% Beginning of the document
\begin{document}
% ####################################################################





% ######################################
% Title of the text
\maketitle
% ######################################





% ######################################
\section{Gładkie funkcje o~zwartym nośniku}

\label{sec:Funkcje-wykladnicze-i-logarytmiczne}
% ######################################



% ##################
\begin{figure}

  \label{fig:Smooth-function-with-compact-support-01}


  \centering

  \begin{tikzpicture}[scale=3]

    % x axis
    \draw[axis arrow] (-2.3,0) -- (2.8,0);

    \pic at (2.8,0) {x mark for horizontal axis 1};


    % y axis
    \draw[axis arrow] (0,-0.5) -- (0,1.9);

    \pic at (0,1.9) {y mark for vertical axis 1};

    \node[right] at (0,0) {$0$};





    % Graph of smooth function with compact support function
    \draw[color=blue] (-2,0) -- (-0.93,0) -- (-0.92,0.004) --
    (-0.91,0.008) -- (-0.9,0.014) -- (-0.89,0.022) -- (-0.88,0.032) --
    (-0.87,0.044) -- (-0.86,0.058) -- (-0.85,0.073) -- (-0.8,0.168) --
    (-0.75,0.276) -- (-0.7,0.382) -- (-0.65,0.481) -- (-0.6,0.569) --
    (-0.55,0.648) -- (-0.5,0.716) -- (-0.45,0.775) -- (-0.4,0.826) --
    (-0.35,0.869) -- (-0.3,0.905) -- (-0.25,0.935) -- (-0.2,0.959) --
    (-0.15,0.977) -- (-0.1,0.989) -- (-0.05,0.997) -- (0,0.999) --
    (0.05,0.997) -- (0.1,0.989) -- (0.15,0.977) -- (0.2,0.959) --
    (0.25,0.935) -- (0.3,0.905) -- (0.35,0.869) -- (0.4,0.826) --
    (0.45,0.775) -- (0.5,0.716) -- (0.55,0.648) -- (0.6,0.569) --
    (0.65,0.481) -- (0.7,0.382) -- (0.75,0.276) -- (0.8,0.168) --
    (0.85,0.073) -- (0.86,0.058) -- (0.87,0.044) -- (0.88,0.032) --
    (0.89,0.022) -- (0.9,0.014) -- (0.91,0.008) -- (0.92,0.004) --
    (0.93,0) -- (2,0);






    % Thicks on x axis
    \pic at (-2,0) {tick horizontal axis thin};

    \node[below] at (-2,0) {$-2$};


    \pic at (-1.5,0) {tick horizontal axis thin 1};

    \node[node scale small 1,below] at (-1.5,0) {$-1.5$};


    \pic at (-1,0) {tick horizontal axis thin};

    \node[below] at (-1,0) {$-1$};


    \pic at (-0.5,0) {tick horizontal axis thin 1};

    \node[node scale small 2,below] at (-0.5,0) {$-0.5$};


    \pic at (0.5,0) {tick horizontal axis thin 1};

    \node[node scale small 2,below] at (0.5,0) {$0.5$};


    \pic at (1,0) {tick horizontal axis thin};

    \node[below] at (1,0) {$1$};


    \pic at (1.5,0) {tick horizontal axis thin 1};

    \node[node scale small 2,below] at (1.5,0) {$1.5$};


    \pic at (2,0) {tick horizontal axis thin};

    \node[below] at (2,0) {$2$};





    % Ticks on y axis
    \pic at (0,-1) {tick vertical axis thin};

    \node[left] at (0,-1) {$-1$};


    \pic at (0,-0.5) {tick vertical axis thin 1};

    \node[node scale small 2,left] at (0,-0.5) {$-0.5$};


    \pic at (0,0.5) {tick vertical axis thin 1};

    \node[node scale small 2,left] at (0,0.5) {$0.5$};


    \pic at (0,1) {tick vertical axis thin};

    \node[left] at (0,1) {$1$};

  \end{tikzpicture}

  \caption{Funkcja $f( x ) = 2.718 * \exp\big( -1/( 1 - x^{ 2 } ) \big)$
    dla~$x \in [ -1, 1 ]$, $f( x ) = 0$ w~pozostałych przypadkach.}


\end{figure}
% ##################





% % ##################
% \begin{figure}

%   \label{fig:Smooth-function-with-compact-support-01}
%   % exp(-1/(1 - x^2)) dla [ -1, 1 ], zero w pozostałych przypadkach.
%   \label{fig:sin-function-01}

%   \centering

%   \begin{tikzpicture}

%     % Graph of smooth function with compact support function






%     % x axis
%     \draw[axis arrow] (-0.3,0) -- (11,0);

%     \pic at (11,0) {x mark for x axis 1};


%     % y axis
%     \draw[axis arrow] (0,-4.75) -- (0,3.2);

%     \pic at (0,3.2) {y mark for y axis 1};



%     \node[number on the right and below] at (0,0) {$0$};





%     % Thicks on x axis
%     \pic at (0.5,0) {tick x axis thin 1};

%     \node[node scale small 2,number below x axis] at (0.5,0) {$0.5$};


%     \pic at (1,0) {tick x axis thin};

%     \node[number below x axis] at (1,0) {$1$};


%     \pic at (1.5,0) {tick x axis thin 1};

%     \node[node scale small 2,number below x axis] at (1.5,0) {$1.5$};


%     \pic at (2,0) {tick x axis thin};

%     \node[below] at (2,0) {$2$};


%     \pic at (2.5,0) {tick x axis thin 1};

%     \node[node scale small 2,below] at (2.5,0) {$2.5$};


%     \pic at (3,0) {tick x axis thin};

%     \node[below] at (3,0) {$3$};


%     \pic at (3.5,0) {tick x axis thin 1};

%     \node[node scale small 2,below] at (3.5,0) {$3.5$};


%     \pic at (4,0) {tick x axis thin};

%     \node[below] at (4,0) {$4$};


%     \pic at (4.5,0) {tick x axis thin 1};

%     \node[node scale small 2,below] at (4.5,0) {$4.5$};


%     \pic at (5,0) {tick x axis thin};

%     \node[below] at (5,0) {$5$};


%     \pic at (5.5,0) {tick x axis thin 1};

%     \node[node scale small 2,below] at (5.5,0) {$5.5$};


%     \pic at (6,0) {tick x axis thin};

%     \node[below] at (6,0) {$6$};


%     \pic at (6.5,0) {tick x axis thin 1};

%     \node[node scale small 2,below] at (6.5,0) {$6.5$};


%     \pic at (7,0) {tick x axis thin};

%     \node[below] at (7,0) {$7$};


%     \pic at (7.5,0) {tick x axis thin 1};

%     \node[node scale small 2,below] at (7.5,0) {$7.5$};


%     \pic at (8,0) {tick x axis thin};

%     \node[below] at (8,0) {$8$};


%     \pic at (8.5,0) {tick x axis thin 1};

%     \node[node scale small 2,below] at (8.5,0) {$8.5$};


%     \pic at (9,0) {tick x axis thin};

%     \node[below] at (9,0) {$9$};


%     \pic at (9.5,0) {tick x axis thin 1};

%     \node[node scale small 2,below] at (9.5,0) {$9.5$};


%     \pic at (10,0) {tick x axis thin};

%     \node[below] at (10,0) {$10$};





%     % Ticks on y axis
%     \pic at (0,-4.5) {tick y axis thin 1};

%     \node[node scale small 2,left] at (0,-4.5) {$-4.5$};


%     \pic at (0,-4) {tick y axis thin};

%     \node[left] at (0,-4) {$4$};


%     \pic at (0,-3.5) {tick y axis thin 1};

%     \node[node scale small 2,left] at (0,-3.5) {$-3.5$};


%     \pic at (0,-3) {tick y axis thin};

%     \node[left] at (0,-3) {$-3$};


%     \pic at (0,-2.5) {tick y axis thin 1};

%     \node[node scale small 2,left] at (0,-2.5) {$-2.5$};


%     \pic at (0,-2) {tick y axis thin};

%     \node[left] at (0,-2) {$-2$};


%     \pic at (0,-1.5) {tick y axis thin 1};

%     \node[node scale small 2,left] at (0,-1.5) {$-1.5$};


%     \pic at (0,-1) {tick y axis thin};

%     \node[number on the left of y axis] at (0,-1) {$-1$};


%     \pic at (0,-0.5) {tick y axis thin 1};

%     \node[node scale small 2,number on the left of y axis] at (0,-0.5)
%     {$-0.5$};


%     \pic at (0,0.5) {tick y axis thin 1};

%     \node[node scale small 2,left] at (0,0.5) {$0.5$};


%     \pic at (0,1) {tick y axis thin};

%     \node[left] at (0,1) {$1$};


%     \pic at (0,1.5) {tick y axis thin 1};

%     \node[node scale small 2,left] at (0,1.5) {$1.5$};


%     \pic at (0,2) {tick y axis thin};

%     \node[left] at (0,2) {$2$};


%     \pic at (0,2.5) {tick y axis thin 1};

%     \node[node scale small 2,left] at (0,2.5) {$2.5$};

%   \end{tikzpicture}

%   \caption{Funkcja $\ln$}


% \end{figure}
% % ##################










% ######################################
\section{Funkcje trygonometryczne}

\label{sec:Funkcje-trygonometryczne}
% ######################################

















% ####################################################################
% ####################################################################
% Bibliography

% \printbibliography





% ############################
% End of the document

\end{document}
