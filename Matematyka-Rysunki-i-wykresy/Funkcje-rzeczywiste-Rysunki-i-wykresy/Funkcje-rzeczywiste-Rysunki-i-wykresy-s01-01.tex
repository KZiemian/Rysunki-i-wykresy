% ------------------------------------------------------------------------------------------------------------------
% Basic configuration and packages
% ------------------------------------------------------------------------------------------------------------------
% Package for discovering wrong and outdated usage of LaTeX.
% More information to be found in l2tabu English version.
\RequirePackage[l2tabu, orthodox]{nag}
% Class of LaTeX document: {size of paper, size of font}[document class]
\documentclass[a4paper,11pt]{article}



% ------------------------------------------------------
% Packages not tied to particular normal language
% ------------------------------------------------------
% This package should improved spaces in the text
\usepackage{microtype}
% Add few important symbols, like text Celcius degree
\usepackage{textcomp}



% ------------------------------------------------------
% Polonization of LaTeX document
% ------------------------------------------------------
% Basic polonization of the text
\usepackage[MeX]{polski}
% Switching on UTF-8 encoding
\usepackage[utf8]{inputenc}
% Adding font Latin Modern
\usepackage{lmodern}
% Package is need for fonts Latin Modern
\usepackage[T1]{fontenc}



% ------------------------------------------------------
% Setting margins
% ------------------------------------------------------
\usepackage[a4paper, total={14cm, 25cm}]{geometry}



% ------------------------------------------------------
% Setting vertical spaces in the text
% ------------------------------------------------------
% Setting space between lines
\renewcommand{\baselinestretch}{1.1}

% Setting space between lines in tables
\renewcommand{\arraystretch}{1.4}



% ------------------------------------------------------
% Packages for scientific papers
% ------------------------------------------------------
% Switching off \lll symbol, that I guess is representing letter "Ł"
% It collide with `amsmath' package's command with the same name
\let\lll\undefined
% Basic package from American Mathematical Society (AMS)
\usepackage[intlimits]{amsmath}
% Equations are numbered separately in every section
\numberwithin{equation}{section}

% Other very useful packages from AMS
\usepackage{amsfonts}
\usepackage{amssymb}
\usepackage{amscd}
\usepackage{amsthm}

% Package with better looking calligraphy fonts
\usepackage{calrsfs}

% Package with better looking greek letters
% Example of use: pi -> \uppi
\usepackage{upgreek}
% Improving look of lambda letter
\let\oldlambda\Lambda
\renewcommand{\lambda}{\uplambda}




% ------------------------------------------------------
% BibLaTeX
% ------------------------------------------------------
% Package biblatex, with biber as its backend, allow us to handle
% bibliography entries that use Unicode symbols outside ASCII
\usepackage[
language=polish,
backend=biber,
style=alphabetic,
url=false,
eprint=true,
]{biblatex}

\addbibresource{Logika-i-teoria-mnogości-Bibliography.bib}





% ------------------------------------------------------
% Defining new environments (?)
% ------------------------------------------------------
% Defining enviroment "Wniosek"
% \newtheorem{corollary}{Wniosek}
% \newtheorem{definition}{Definicja}
% \newtheorem{theorem}{Twierdzenie}





% ------------------------------------------------------
% Wonderful package PGF/TikZ
% ------------------------------------------------------
\usepackage{tikz}

% Loding TikZ libraries

% Library for positioning nodes
\usetikzlibrary{positioning}

% Styles for arrows
\usepackage{./Local-packages/PGF-TikZ-Arrows-styles}

% Node and pics for drawing charts
\usepackage{./Local-packages/PGF-TikZ-Chart-nodes-and-pics}




% ------------------------------------------------------
% Local packages
% You need to put them in the same directory as .tex file
% ------------------------------------------------------
% Package containing various command useful for working with a text
% \usepackage{general-commands}
% Package containing commands and other code useful for working with
% mathematical text
% \usepackage{math-commands}





% ------------------------------------------------------
% Package "hyperref"
% They advised to put it on the end of preambule
% ------------------------------------------------------
% It allows you to use hyperlinks in the text
\usepackage{hyperref}










% ------------------------------------------------------------------------------------------------------------------
% Title and author of the text
\title{Funkcje rzeczywiste \\
  {\Large Rysunki i~wykresy}}

\author{Kamil Ziemian}


% \date{}
% ------------------------------------------------------------------------------------------------------------------










% ####################################################################
% Beginning of the document
\begin{document}
% ####################################################################





% ######################################
% Title of the text
\maketitle
% ######################################





% ######################################
\section{Gładkie funkcje o~zwartym nośniku}

\label{sec:Funkcje-wykladnicze-i-logarytmiczne}
% ######################################



% ##################
\begin{figure}
  % !!!!!!!!!!!!!!!!!!!!!!!!!!!!!!
  \label{fig:Smooth-function-with-compact-support-01}
  % 2.718 * exp(-1/(1 - x^2)) dla [ -1, 1 ], zero w pozostałych przypadkach.


  \centering

  \begin{tikzpicture}

    % x axis
    \draw[axis arrow] (-2.3,0) -- (2.8,0);

    \pic at (2.8,0) {x mark for horizontal axis 1};


    % y axis
    \draw[axis arrow] (0,-0.5) -- (0,1.9);

    \pic at (0,1.9) {y mark for vertical axis 1};

    \node[right] at (0,0) {$0$};





    % Graph of smooth function with compact support function
    \draw[color=blue] (-2,0) -- (-0.93,0) -- (-0.92,0.004) --
    (-0.91,0.008) -- (-0.9,0.014) -- (-0.89,0.022) -- (-0.88,0.032) --
    (-0.87,0.044) -- (-0.86,0.058) -- (-0.85,0.073) -- (-0.8,0.168) --
    (-0.75,0.276) -- (-0.7,0.382) -- (-0.65,0.481) -- (-0.6,0.569) --
    (-0.55,0.648) -- (-0.5,0.716) -- (-0.45,0.775) -- (-0.4,0.826) --
    (-0.35,0.869) -- (-0.3,0.905) -- (-0.25,0.935) -- (-0.2,0.959) --
    (-0.15,0.977) -- (-0.1,0.989) -- (-0.05,0.997) -- (0,0.999) --
    (0.05,0.997) -- (0.1,0.989) -- (0.15,0.977) -- (0.2,0.959) --
    (0.25,0.935) -- (0.3,0.905) -- (0.35,0.869) -- (0.4,0.826) --
    (0.45,0.775) -- (0.5,0.716) -- (0.55,0.648) -- (0.6,0.569) --
    (0.65,0.481) -- (0.7,0.382) -- (0.75,0.276) -- (0.8,0.168) --
    (0.85,0.073) -- (0.86,0.058) -- (0.87,0.044) -- (0.88,0.032) --
    (0.89,0.022) -- (0.9,0.014) -- (0.91,0.008) -- (0.92,0.004) --
    (0.93,0) -- (2,0);






    % Thicks on x axis
    % \pic at (-2.5,0) {tick x axis thin 1};

    % \node[node scale small 1,below] at (-2.5,0) {$-2.5$};


    \pic at (-2,0) {tick x axis thin};

    \node[below] at (-2,0) {$-2$};


    \pic at (-1.5,0) {tick x axis thin 1};

    \node[node scale small 1,below] at (-1.5,0) {$-1.5$};


    \pic at (-1,0) {tick x axis thin};

    \node[below] at (-1,0) {$-1$};


    \pic at (-0.5,0) {tick x axis thin 1};

    \node[node scale small 2,number below x axis] at (-0.5,0) {$-0.5$};


    \pic at (0.5,0) {tick x axis thin 1};

    \node[node scale small 2,number below x axis] at (0.5,0) {$0.5$};


    \pic at (1,0) {tick x axis thin};

    \node[number below x axis] at (1,0) {$1$};


    \pic at (1.5,0) {tick x axis thin 1};

    \node[node scale small 2,number below x axis] at (1.5,0) {$1.5$};


    \pic at (2,0) {tick x axis thin};

    \node[below] at (2,0) {$2$};





    % Ticks on y axis
    % \pic at (0,-1) {tick y axis thin};

    % \node[left] at (0,-1) {$-1$};


    % \pic at (0,-0.5) {tick y axis thin 1};

    % \node[node scale small 2,left] at (0,-0.5) {$-0.5$};


    \pic at (0,0.5) {tick y axis thin 1};

    \node[node scale small 2,left] at (0,0.5) {$0.5$};


    \pic at (0,1) {tick y axis thin};

    \node[left] at (0,1) {$1$};

  \end{tikzpicture}

  \caption{Funkcja $\sin$}


\end{figure}
% ##################





% % ##################
% \begin{figure}

%   \label{fig:Smooth-function-with-compact-support-01}
%   % exp(-1/(1 - x^2)) dla [ -1, 1 ], zero w pozostałych przypadkach.
%   \label{fig:sin-function-01}

%   \centering

%   \begin{tikzpicture}

%     % Graph of smooth function with compact support function
%     % \draw[color=blue] (0.01,-4.605) -- (0.05,-2.995) -- (0.1,-2.2302) --
%     % (0.15,-1.897) -- (0.2,-1.609) -- (0.25,-1.386) -- (0.3,-1.203) --
%     % (0.35,-1.049) -- (0.4,-0.916) -- (0.45,-0.798) -- (0.5,-0.693) --
%     % (0.55,-0.597) -- (0.6,-0.51) -- (0.65,-0.43) -- (0.7,-0.356) --
%     % (0.75,-0.287) -- (0.8,-0.223) -- (0.85,-0.162) -- (0.9,-0.105) --
%     % (0.95,-0.051) -- (1,0) -- (1.05,0.048) -- (1.1,0.095) --
%     % (1.15,0.139) -- (1.2,0.182) -- (1.25,0.223) -- (1.3,0.262) --
%     % (1.35,0.3) -- (1.4,0.336) -- (1.45,0.371) -- (1.5,0.405) --
%     % (1.55,0.438) -- (1.6,0.47) -- (1.65,0.5) -- (1.7,0.53) --
%     % (1.75,0.559) -- (1.8,0.587) -- (1.85,0.615) -- (1.9,0.641) --
%     % (1.95,0.667) -- (2,0.693) -- (2.05,0.717) -- (2.1,0.741) --
%     % (2.15,0.765) -- (2.2,0.788) -- (2.25,0.81) -- (2.3,0.832) --
%     % (2.35,0.854) -- (2.4,0.875) -- (2.45,0.896) -- (2.5,0.912) --
%     % (2.55,0.936) -- (2.6,0.955) -- (2.65,0.974) -- (2.7,0.993) --
%     % (2.75,1.011) -- (2.8,1.029) -- (2.85,1.047) -- (2.9,1.064) --
%     % (2.95,1.081) -- (3,1.098) -- (3.05,1.115) -- (3.1,1.131) --
%     % (3.15,1.147) -- (3.2,1.163) -- (3.25,1.178) -- (3.3,1.193) --
%     % (3.35,1.208) -- (3.4,1.223) -- (3.45,1.238) -- (3.5,1.252) --
%     % (3.55,1.266) -- (3.6,1.28) -- (3.65,1.29) -- (3.7,1.308) --
%     % (3.75,1.321) -- (3.8,1.335) -- (3.85,1.348) -- (3.9,1.36) --
%     % (3.95,1.373) -- (4,1.386) -- (4.05,1.398) -- (4.1,1.41) --
%     % (4.15,1.423) -- (4.2,1.435) -- (4.25,1.446) -- (4.3,1.458) --
%     % (4.35,1.47) -- (4.4,1.481) -- (4.45,1.492) -- (4.5,1.504) --
%     % (4.55,1.515) -- (4.6,1.526) -- (4.65,1.536) -- (4.7,1.547) --
%     % (4.75,1.558) -- (4.8,1.568) -- (4.85,1.578) -- (4.9,1.589) --
%     % (4.95,1.599) -- (5,1.609) -- (5.05,1.619) -- (5.1,1.629) --
%     % (5.15,1.638) -- (5.2,1.648) -- (5.25,1.658) -- (5.3,1.667) --
%     % (5.35,1.677) -- (5.4,1.686) -- (5.45,1.695) -- (5.5,1.704) --
%     % (5.55,1.713) -- (5.6,1.722) -- (5.65,1.731) -- (5.7,1.74) --
%     % (5.75,1.749) -- (5.8,1.757) -- (5.85,1.766) -- (5.9,1.774) --
%     % (5.95,1.783) -- (6,1.791) -- (6.05,1.8) -- (6.1,1.808) --
%     % (6.15,1.816) -- (6.2,1.824) -- (6.25,1.832) -- (6.3,1.84) --
%     % (6.35,1.848) -- (6.4,1.856) -- (6.45,1.864) -- (6.5,1.871) --
%     % (6.55,1.879) -- (6.6,1.887) -- (6.65,1.894) -- (6.7,1.902) --
%     % (6.75,1.909) -- (6.8,1.916) -- (6.85,1.924) -- (6.9,1.931) --
%     % (6.95,1.938) -- (7,1.945) -- (7.05,1.953) -- (7.1,1.96) --
%     % (7.15,1.967) -- (7.2,1.974) -- (7.25,1.981) -- (7.3,1.987) --
%     % (7.35,1.994) -- (7.4,2.001) -- (7.45,2.008) -- (7.5,2.014) --
%     % (7.55,2.021) -- (7.6,2.028) -- (7.65,2.034) -- (7.7,2.041) --
%     % (7.75,2.047) -- (7.8,2.054) -- (7.85,2.06) -- (7.9,2.066) --
%     % (7.95,2.073) -- (8,2.079) -- (8.05,2.085) -- (8.1,2.091) --
%     % (8.15,2.098) -- (8.2,2.104) -- (8.25,2.11) -- (8.3,2.116) --
%     % (8.35,2.122) -- (8.4,2.128) -- (8.45,2.134) -- (8.5,2.14) --
%     % (8.55,2.145) -- (8.6,2.151) -- (8.65,2.157) -- (8.7,2.163) --
%     % (8.75,2.169) -- (8.8,2.17) -- (8.85,2.18) -- (8.9,2.186) --
%     % (8.95,2.191) -- (9,2.197) -- (9.05,2.202) -- (9.1,2.208) --
%     % (9.15,2.213) -- (9.2,2.219) -- (9.25,2.224) -- (9.3,2.23) --
%     % (9.35,2.235) -- (9.4,2.24) -- (9.45,2.246) -- (9.5,2.251) --
%     % (9.55,2.256) -- (9.6,2.261) -- (9.65,2.266) -- (9.7,2.272) --
%     % (9.75,2.277) -- (9.8,2.282) -- (9.85,2.287) -- (9.9,2.292) --
%     % (9.95,2.297) -- (10,2.302);





%     % x axis
%     \draw[axis arrow] (-0.3,0) -- (11,0);

%     \pic at (11,0) {x mark for x axis 1};


%     % y axis
%     \draw[axis arrow] (0,-4.75) -- (0,3.2);

%     \pic at (0,3.2) {y mark for y axis 1};



%     \node[number on the right and below] at (0,0) {$0$};





%     % Thicks on x axis
%     \pic at (0.5,0) {tick x axis thin 1};

%     \node[node scale small 2,number below x axis] at (0.5,0) {$0.5$};


%     \pic at (1,0) {tick x axis thin};

%     \node[number below x axis] at (1,0) {$1$};


%     \pic at (1.5,0) {tick x axis thin 1};

%     \node[node scale small 2,number below x axis] at (1.5,0) {$1.5$};


%     \pic at (2,0) {tick x axis thin};

%     \node[below] at (2,0) {$2$};


%     \pic at (2.5,0) {tick x axis thin 1};

%     \node[node scale small 2,below] at (2.5,0) {$2.5$};


%     \pic at (3,0) {tick x axis thin};

%     \node[below] at (3,0) {$3$};


%     \pic at (3.5,0) {tick x axis thin 1};

%     \node[node scale small 2,below] at (3.5,0) {$3.5$};


%     \pic at (4,0) {tick x axis thin};

%     \node[below] at (4,0) {$4$};


%     \pic at (4.5,0) {tick x axis thin 1};

%     \node[node scale small 2,below] at (4.5,0) {$4.5$};


%     \pic at (5,0) {tick x axis thin};

%     \node[below] at (5,0) {$5$};


%     \pic at (5.5,0) {tick x axis thin 1};

%     \node[node scale small 2,below] at (5.5,0) {$5.5$};


%     \pic at (6,0) {tick x axis thin};

%     \node[below] at (6,0) {$6$};


%     \pic at (6.5,0) {tick x axis thin 1};

%     \node[node scale small 2,below] at (6.5,0) {$6.5$};


%     \pic at (7,0) {tick x axis thin};

%     \node[below] at (7,0) {$7$};


%     \pic at (7.5,0) {tick x axis thin 1};

%     \node[node scale small 2,below] at (7.5,0) {$7.5$};


%     \pic at (8,0) {tick x axis thin};

%     \node[below] at (8,0) {$8$};


%     \pic at (8.5,0) {tick x axis thin 1};

%     \node[node scale small 2,below] at (8.5,0) {$8.5$};


%     \pic at (9,0) {tick x axis thin};

%     \node[below] at (9,0) {$9$};


%     \pic at (9.5,0) {tick x axis thin 1};

%     \node[node scale small 2,below] at (9.5,0) {$9.5$};


%     \pic at (10,0) {tick x axis thin};

%     \node[below] at (10,0) {$10$};





%     % Ticks on y axis
%     \pic at (0,-4.5) {tick y axis thin 1};

%     \node[node scale small 2,left] at (0,-4.5) {$-4.5$};


%     \pic at (0,-4) {tick y axis thin};

%     \node[left] at (0,-4) {$4$};


%     \pic at (0,-3.5) {tick y axis thin 1};

%     \node[node scale small 2,left] at (0,-3.5) {$-3.5$};


%     \pic at (0,-3) {tick y axis thin};

%     \node[left] at (0,-3) {$-3$};


%     \pic at (0,-2.5) {tick y axis thin 1};

%     \node[node scale small 2,left] at (0,-2.5) {$-2.5$};


%     \pic at (0,-2) {tick y axis thin};

%     \node[left] at (0,-2) {$-2$};


%     \pic at (0,-1.5) {tick y axis thin 1};

%     \node[node scale small 2,left] at (0,-1.5) {$-1.5$};


%     \pic at (0,-1) {tick y axis thin};

%     \node[number on the left of y axis] at (0,-1) {$-1$};


%     \pic at (0,-0.5) {tick y axis thin 1};

%     \node[node scale small 2,number on the left of y axis] at (0,-0.5)
%     {$-0.5$};


%     \pic at (0,0.5) {tick y axis thin 1};

%     \node[node scale small 2,left] at (0,0.5) {$0.5$};


%     \pic at (0,1) {tick y axis thin};

%     \node[left] at (0,1) {$1$};


%     \pic at (0,1.5) {tick y axis thin 1};

%     \node[node scale small 2,left] at (0,1.5) {$1.5$};


%     \pic at (0,2) {tick y axis thin};

%     \node[left] at (0,2) {$2$};


%     \pic at (0,2.5) {tick y axis thin 1};

%     \node[node scale small 2,left] at (0,2.5) {$2.5$};

%   \end{tikzpicture}

%   \caption{Funkcja $\ln$}


% \end{figure}
% % ##################










% ######################################
\section{Funkcje trygonometryczne}

\label{sec:Funkcje-trygonometryczne}
% ######################################

















% ####################################################################
% ####################################################################
% Bibliography

\printbibliography





% ############################
% End of the document

\end{document}
